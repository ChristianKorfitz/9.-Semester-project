\section{Statistical analysis} \label{stats}
The purpose of the statistical analysis was to detect the areas of activation associated with the noxious heat stimuli of 48$\degree$. A statistical analysis was performed to show statistically significant activation within single heat runs, within single participants and within groups. General linear models (GLM) were used in all analyses. A more detailed explanation of the GLM can be found in the appendix \ref{back:sec:glm}. 

\subsection{Within heat run analysis}
On a participant level the aim of the fMRI analysis was to analyze the time courses of every voxel to find out if the BOLD signal changed in response to some perturbation. In the context of this project, we were interested in finding the response to the repeated block designed heat stimulus delivered to the participants. GLM was used for analysing this through the FSL tool FEAT (fMRIB’s Expert Analysis Tool) using a multivariate linear regression approach, as shown in \figref{fig:GLM}, \secref{back:sec:glm}. The BOLD signal time courses made up the dependent variable and the expected BOLD signal response for the 48$\degree$ stimuli, 47\degree stimuli and the rating period made up the independent variables. As only the BOLD signal response of the 48\degree stimuli was the signal of interest, the 47\degree stimuli and the rating period BOLD signal response regressors were set as nuisance regressors. The nuisance regressors are used to account for the variability in the BOLD signal response of no interest. \\
The BOLD signal response is not instantaneous with the delivered stimuli, but delayed and blurred as a result of the hemodynamic response. It was therefore modeled by convoluting each of the regressors with a hemodynamic response function (HRF). As a default function FEAT uses a gamma function, which was also applied for this project. \\
The intercept and estimation parameters were estimated through FEAT by applying the ordinary least square method that minimizes distances between dependent and independent variables. The estimation parameters from the regressor of interest for each voxel were used to test the hypothesis about whether the activation was associated with the 48\degree stimuli. A t-test was applied to test if the estimation parameters were significantly different from 0, thus showing an effect. This resulted in a spatial map of activation for a single heat run for one participant, along with the within participant variance $\sigma_{W}^{2}$. This process was performed for every heat run for each participant on the data from standard preprocessing and FIX preprocessing, respectively. \\

\subsection{Within participant analysis}
To combine the results obtained for the individual heat runs for a single participant a fixed effect model was used. This fixed effect model amounted a weighted average of the single heat run effect. This was done through the FIXED algorithm in FEAT that uses a weighted linear regression. Here the weights were the inverse of the within participant variances $\sigma_{W}^{2}$ estimated for each of the three heat runs and the input was the spatial maps obtained. This was performed for both the standard preprocessed data and the FIX preprocessed data, providing a full spatial map of each subject for both preprocessing methods.

\subsection{Within group analysis}
The aim of the group-level analysis was to combine the results obtained from each participant. This was done by using a mixed effect model, which treats the group as a random sample from a total population. This mixed effect model uses an addition of the within participant variance $\sigma_{W}^{2}$ and the between participant variance $\sigma_{B}^{2}$ in a weighted linear regression approach, which takes the spatial maps estimated in the fixed effect model as input for the independent variable. The within participant variances were estimated in the participant-level analysis, while $\sigma_{B}^{2}$ was estimated through the FLAME (FMRIB's Local Analysis of Mixed Effects) algorithm in the mixed effect model option in FEAT, which uses an iterative process for the estimation. This is all done on a voxel level to provide more sensitive estimations.
To achieve a full spatial map of all subjects combined for both the standard preprocessed data and the FIX preprocessed data respectively, a t-test was used to test if the estimated parameters differed significantly from 0 for the data from each of the two preprocessing methods respectively. \\
To test the difference in activation in the data of the two preprocessing methods, the contrast of the hypothesis was simply changed. This provided two spatial maps: one where the activation was significantly higher in the FIX preprocessed data compared to the standard preprocessed data, and one where the activation was significantly higher in the standard preprocessed data compared to the FIX preprocessed data. 
