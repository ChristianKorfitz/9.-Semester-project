The purpose of the statistical analyses was to detect the areas of activation associated with the noxious heat stimuli of 48$\degree$. A statistical analysis was performed to show statistically significant activation within single heat runs, within single participants and within groups, respectively. General linear models (GLM) were used in all the analyses. A more detailed explanation of the GLM can be found in the appendix \ref{Appendix}. All analyses were performed on the data from standard preprocessing and FIX preprocessing, respectively.

\subsection{Within heat run analysis}
At a single heat run level the aim of the fMRI analysis was to analyze the time courses of every voxel to find out if the BOLD signal changed in response to some perturbation. In this project, the aim of this analysis was to define brain responses to the 48$\degree$ heat stimuli delivered to the participants. GLM was performed with the FSL tool FEAT (fMRIB’s Expert Analysis Tool) using a multivariate linear regression approach, as shown in \figref{fig:Appendix}. The BOLD signal time courses constituted the dependent variable, while the expected BOLD signal response for the 48$\degree$ stimuli, 47$\degree$ stimuli and the rating period made up the independent variables. As only the BOLD signal response of the 48$\degree$ stimuli was the signal of interest, the expected 47$\degree$ stimuli and the expected signal response to the rating periods were set as nuisance regressors. The nuisance regressors were used to account for the variability in the BOLD signal response of no interest and to keep a clean baseline. \\
The BOLD signal response is not instantaneous with the delivered stimulus, but delayed after the stimulus as a result of the hemodynamic response. It was therefore modeled by convoluting each regressor with a hemodynamic response function (HRF). The default shape of the HFT in FEAT is a  gamma function, which was also selected for this project. \\
The intercept and parameters estimated were calculated through FEAT by applying the ordinary least square method that minimizes distances between dependent and independent variables. The parameters estimates from the regressor of interest for each voxel were used to test the hypothesis about whether changes in brain activation was associated with the regressor of interest, i.e. 48$\degree$ stimuli. A t-test was applied to test if the parameters estimates were significantly different from the baseline. This resulted in a spatial map of significant activation for each heat run and participant, along with the within participant variance $\sigma_{W}^{2}$. \\
The parameter estimates explaining the activation were transformed into z-statistics, from which a spatial map of activation explained in z-scores was constructed. Furthermore a thresholding was applied, so that only clusters of activation above a z-score of 2.3 was kept. This transformation was also applied on the parameter estimates obtained in the following two levels of analysis to construct the spatial maps of activation. 

\subsection{Within participant analysis}
To combine the results obtained for the individual heat runs for a single participant a fixed effect model was used. This fixed effect model applied a weighted average of the single heat run effect through a weighted linear regression approach. This was performed with the FIXED algorithm in FEAT. Here the weights were the inverse of the within participant variances $\sigma_{W}^{2}$ estimated for each of the three heat runs and the independent variable inputs were the parameter estimates acquired from the previous analysis. This analysis resulted in a new set of parameter and variance estimates, providing a single spatial map of significant activation for each participant for both preprocessing methods, respectively.

\subsection{Within group analysis}
The aim of the group-level analysis was to combine the results obtained from each participant. This was done by using a mixed effect model, which treats the group as a random sample from a total population. This mixed effect model uses the sum of the within participant variance $\sigma_{W}^{2}$ and the between participant variance $\sigma_{B}^{2}$ as weights in a weighted linear regression approach. The parameters estimated in the fixed effect model were taken as the independent variable input. The within participant variances were estimated in the participant-level analysis, while $\sigma_{B}^{2}$ was estimated through the FLAME (FMRIB's Local Analysis of Mixed Effects) algorithm in the mixed effect model option in FEAT, which uses an iterative process for the estimation. This is all done on a voxel level to provide more sensitive estimations.
To achieve a full spatial map of activation for all participants combined for both the standard preprocessed data and the FIX preprocessed data, respectively, a t-test was used to test if the estimated parameters differed significantly from the baseline for the data from each of the two preprocessing methods, respectively. 
To test the difference in activation in the data of the two preprocessing methods, the contrast of the hypothesis was simply changed. This provided two spatial maps: one where the activation was significantly higher in the FIX preprocessed data compared to the standard preprocessed data, and one where the activation was significantly higher in the standard preprocessed data compared to the FIX preprocessed data. \\
To address the secondary research question it was analyzed if there was a relation between the pain intensity ratings reported by the participants and the associated brain activation. To do this the mean pain intensity ratings for each participant were set as independent variables in a within group analysis, and the within participant brain activation for each participant were set the dependent variable. The regression model then estimated parameters for each voxel. If that parameter was not significantly higher than a z-score of 2.3 when transformed into z-statistics, no association between pain intensity rating and brain activation would be shown. 

\subsection{Activation distribution}
Another way of comparing the two preprocessing methods is by looking at what activated brain regions different amounts of the participant population have in common. To do this first the spatial maps of all participants were merged. Subsequently, the two merged spatial maps were binarized, leaving all activated voxels as 1 and all non-activated voxels as 0. Lastly, the mean across time was calculated for the two spatial maps, respectively. 
By superimposing the two spatial maps,
