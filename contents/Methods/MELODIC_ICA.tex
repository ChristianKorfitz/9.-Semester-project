\subsection{MELODIC ICA}

\fxnote{NEEDS TO BE DOCUMENTED}


%Independent component analysis (ICA) – one of the most widely used techniques for the exploratory analysis offMRI data – has shown to be a powerful technique in iden- tifying various sources ofneuronally-related and artefactual fluctuation in fMRI data (bothwith the application of external stimuli and with the subject “at rest”). ICAdecomposes fMRI data into patterns ofactivity (a set ofspatial maps and their corresponding time series) that are statistically independent and add linearly to explain voxel- wise time series. Given the set of ICA components, if the components representing “signal” (brain activity) can be distinguished form the “noise” components (effects of motion, non-neuronal physiology, scanner artefacts and other nuisance sources), the latter can then be removed from the data. ). Thismodels the data as a linearmix- ture of different processes, the spatial distributions ofwhich are time- invariant (apart from overall amplitude modulation by the associated timecourse) and statistically independent. ICA assumes the following linear model Y ¼ AMþ E where Y is the T × Vmatrix offMRI time series with Ttime samples and Vvoxels; M is a K × Vmatrix ofK≪ T spatial components of the inde- pendent sources (comprisingVvoxels each) andA is the T × Kmatrix of the Kcorresponding time courses (comprising T samples each). E is the residuals in the probabilistic ICA model (Beckmann and Smith, 2004), and is assumed to comprise the unstructured noise that dominates the weakest eigenvectors of an initial principal components analysis de- composition applied before the main ICA algorithm. To reduce the structured noise using ICA, it is necessary to identi-fy the subset ofA andM that demonstrate artefactual behaviour tem- porally and/or spatially. Having found such a subset, one can clean the data by (for example) regressing the set of artefactual time courses Ab out of the original data, or by taking the product of arte- fact time courses and spatial maps AbMb and subtracting that from the data. For detailed investigations of different methods for re- gressing the artefactual components out of the data, see Griffanti et