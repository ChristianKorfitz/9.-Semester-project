\subsection{Hand labeling of independent components}
Before the automatic FIX classifier could be used to identity components of noise and interest it needed to be trained with a set of hand labeled components \cite{Salimi-Khorshidi2014}. For this purpose 25 components were extracted from all heat runs for 34 participants using the MELODIC tool, after which each component was manually hand labeled. \\
The process of the hand labeling followed that the investigators labeled half of the components each, where after the labeled components were cross-checked and discussed by the investigators. The investigator assessment relied heavily on information with regards to hand labeling provided by Griffanti et al. \cite{Griffanti2017} and Salimi-Khorshidi et al. \cite{Salimi-Khorshidi2014}. As a final assessment an expert in the field of fMRI did a control and corrected labels if they not did not fit the expert opinion. \\
The MELODIC program proposed a set of predefined classes to label the components as. The classes suited for artifacts were movement, susceptibility-motion, MRI, cardiac, respiratory, white matter, sagittal sinus and unclassified noise. It was possible that one component could contain more than one artefact class. If all artefact classes of the component were identifiable, more classes were added to the component. The unclassified noise class was proposed for components containing a mixture of different artifact sources, which were less clear to pinpoint. A class named unknown was intended for labeling components that contained both noise and signal. For any signal of interest the signal class was proposed, as there was no discrimination between different types of signal of interest. 
The following illustrations\fxnote{ILLUSTRATIONS ARE NOT INCLUDED YET} show transverse spatial examples of components related to the various artifact classes, the signal class and the unknown class. It should be noted that the coronal and sagittal views were additionally available, along with the time course and frequency content of the components for shaping the assessment of the labeling. The corresponding structural images were used to underlay the components to assess in which brain regions the components were located.\fxnote{WE ARE THINKING OF MAKING A ACTIVITY DIAGRAM SHOWING THE REASONING FOR WHICH LABEL, A COMPONENT WOULD BE GIVEN}

%movement. \caption{An example component of the movement artefact. Movement noise will usually be seen as a ring or lines on the periphery of the brain. Depending on the nature of the movement causing the noise, the voxels will be located on the inside or outside of the brain. r 94.2.8}
%
%susceptibility-motion \caption{An example component of the susceptibility-motion artefact. Susceptibility-motion noise is mainly due to air-tissue interfaces, and thus usually spatially expressed as a mix of activation and deactivation at the sinuses of the skull. 78.3.18}
%
%MRI \caption{An example component of the MRI artefact. This artefact is usually visualized as fully activated (or deactivated) single slices in separated temporal levels. In the sagittal or coronal plane, this is reflected as clear stripes in the images, which can not be explained as physiologically meaningful. Another example is slices containing large portions of alternating activation and deactivation, which too can not be physiologically explained. 146.1.1}
%
%cardiac \caption{An example component of cardiac and respiratory artefacts. Cerebrospinal fluid pulsation is usually due to respiratory and cardiac cycles. Thus, activation of the ventricular system of the brain will be an indicator of respiratory and/or cardiac artefacts. 94.3.16}
%
%white matter \caption{An example component of the white matter artefact. White matter artefacts are visible as activation in the white matter area of the brain, while no or low activation of areas of interest of noise are visible. 169.3.23}
%
%sagittal sinus \caption{An example component of the sagittal sinus artefact. Sagittal sinus artefacts are visible as activation of the sagittal sinus veins, while no or low activation of areas of interest or noise are visible. 94.2.2}
%
%unclassified noise \caption{An example component of unclassified noise.196.2.17}
%
%unknown \caption{An example component of a mix of signal of interest and noise.100.1.11}
%
%signal \caption{An example component of signal of interest. 94.1.3}
