\subsection{Feature extraction}
Besides training the classifier with the label of each component, it was trained with features extracted from each component. An important step in establishing a robust classification model is extracting effective features to feed the classifier. Extraction of reasonably independent features that correlate with the targeted class will establish a more robust classifier when training it. In the FIX algorithm 186 features with temporal or spatial characteristics were extracted. The first feature extracted was, however, considered of spatial-temporal origin and consisted of the number of ICs estimated by MELODIC. It would be expected that the noise present in the data would affect the number of ICs extracted. \cite{Salimi-Khorshidi2014} However, in this study the number of ICs calculated for each heat run was fixed on 25, and that feature would therefore have no influence on the expectation of a component containing noise or signal. 
The following sections aim to explain the idea behind extracting these exact temporal and spatial features and to give examples of selected features. Thus, not all 186 features will be covered in depth. For a further explanation of the features we direct the reader to the 2014 study by Salimi-Khorshidi et al. \cite{Salimi-Khorshidi2014} The result of the feature extraction from one heat run consisted of a 25 x 186 matrix - one feature value for each feature of each component.

\subsubsection{Temporal features}
\textit{Autoregressive models} Some temporal features extracted were based on autoregressive (AR) properties of the time course. It was expected that the temporal smoothness derived from AR models, could help distinguish signal from particular artefact components. For an AR(n) model, where $n$ is the order up to $n = 6$, $a_k$ denotes the AR parameters, $e_p$ denotes the variance of the residual of up to AR(6). The first AR property features derived were the slope and intercept of the straight line which explained $e_p$ as a function of $n$, where an increase in $n$ would output a better fit and thus a smaller residual variance. It was expected that an improvement in goodness of fit would decrease as more noise was present in the time course. 
The other AR features extracted were $a_1$, $a_2$, $e_1$ and $e_2$. These features encapsulated the autocorrelation estimated from the low order AR models. It was expected that components containing signal would have a higher temporal autocorrelation and lower residual variance compared to components containing noise. \\
\textit{Distribution information} It was expected that signal components would be normally distributed, while noise components would vary in how they were distributed (e.g. long-tailed distribution and multi-modal), e.g. due to sudden spikes in the time course caused by rapid movements and/or scanner artefacts. Features that gave information on distribution were therefore extracted. These included kurtosis, skewness, mean-median difference, entropy and negentropy. \\
\textit{Jump amplitudes} The extent of jumps in the amplitude of the time course is an important feature in distinguishing signal from noise. Signal components were expected to be fairly smooth, while noise components usually would contain large jump and fluctuations. To describe this characteristic features based on division calculations that include the standard deviation (std), mean or maximum of the differential of the time course and the std, mean and/or maximum of the time course were extracted. \\
\textit{Fourier transform} Taking the fast Fourier transform (fft) of the time course will capture the frequency content, which can be utilized to differentiate between signal and noise. It was expected for signal components to almost exclusively have high power amplitude in low frequencies due to the block design of the stimuli task. Noise components might on the other hand have contained frequencies of the whole spectrum. Therefore several fft-based features were extracted. Examples of features derived were the total power above 0.1, 0.15, 0.2 and 0.25 Hz respectively and the percentage of power that lied in the following intervals: 0:0.01, 0.01:0.025, 0.025:0.05, 0.05:0.1, 0.1:0.15, 0.15:0.2 and 0.2:0.25 Hz. \\
\textit{Correlation} It was expected that time courses of signal components would be strongly associated with grey matter, while noise components would correlate with time courses of white matter and CSF along with the motion correction time courses calculated in the initial preprocessing. The correlation-based features extracted were therefore based on the correlation between the component time course and reference grey matter-, white matter-, CSF-derived time courses and the motion correction time courses. The reference time courses were calculated using FSL’s tissue segmentation tool, from which grey matter, white matter and CSF masks were extracted. Each tissue type's time course was then computed as the average of the time courses that corresponded to that given tissue. \cite{Salimi-Khorshidi2014}

\subsubsection{Spatial features}
\textit{Clusters’ size and spatial distribution} The distribution of activated and deactivated cluster sizes is an important indicator of a component containing either noise or signal. It was expected that signal components contained a small number of relatively large clusters, while noise components contained a large number of small clusters. To capture this information, a list of a spatial map’s clusters $\mathbf{c}$ was formed, where only clusters of at least 5 connected activated or deactivated voxels were kept, and listed in descending order. Feature extracted to summarize $\mathbf{c}$ were: length($\mathbf{c}$), mean($\mathbf{c}$), median($\mathbf{c}$), max($\mathbf{c}$), var($\mathbf{c}$), skewness($\mathbf{c}$), kurtosis($\mathbf{c}$), $\mathbf{c}$[1], $\mathbf{c}$[2], and $\mathbf{c}$[3], where the last three features were the first to third elements of $\mathbf{c}$.\\
Looking at the distribution of clusters in individual slices can help detect scanner artefacts. $\mathbf{V}$ and $\mathbf{U}$ contained slice specific information of the ICA spatial map $\mathbf{m}$ and slice specific information of voxels above 2.5 z-score $\mathbf{m^{\tau}_{p}}$, respectively. Further features consisted of max($\mathbf{V}$) and max($\mathbf{u}$) and slices containing above 15 clusters.\\ % of the total variance of $\mathbf{V}$ and $\mathbf{u}$. Similar features examines the difference in variance explained by even and odd slices as well and the difference in variance explained by slices $[1,2,5,6,9,10, …]$ and $[3,4,7,8,11,12, …]$ for both $\mathbf{m}$ and $\mathbf{m^{\tau}_{p}}$. 
It is not very likely for signal components to have large contents of both activation and deactivation in the spatial map. For this reason features that use the mean, standard deviation and entropy of $\mathbf{m}$ to measure the amount of positive and negative voxels in the components were extracted. \\
\textit{Voxels overlaying dark/bright raw data voxels}
Signal of interest is associated with dark voxels where noise is usually found in bright voxels. The next features were based on multiplying and dividing the components’ spatial map with the mean of the corresponding preprocessed time courses, which formed two new images. It was expected that the intensity information contained information on which voxels were of interest and which were not. The features consisted of the 95th and 99th percentile of the new images. \\
\textit{Percent on brain boundary} High activation/deactivation in the spatial map that overlaps the boundary of the brain and the area outside the brain will most likely be movement related. Segmenting the brain using the FSL BET tool and subtracting this mask with and eroded version of itself will output a mask of the brain’s edge. Five masks with varying thickness were extracted. The features measured how large a percentage of activation in the components spatial maps that was embedded in the edges and how large a percentage of edges was covered by the components activation. The higher these values were, the higher probability the component was to have captured a movement artefact. \\
\textit{Mask-based features} It might be needed to use spatially-specific masks to detect certain noise sources (Sagittal sinus, CSF and White matter) more precisely. The spatial appearance of the major veins could for instance be mistaken for signal of interest and the cluster-like activation pattern may look similar to signal of interest. To detect these structures more conservatively FIX utilized three standard-space masks that consisted of three major veins. From the three masks three different masks with varying thickness were derived, due to the anatomical variability between subjects. The same features as in the brain boundary features were extracted, where these new masks were utilized instead. The same procedure was executed for grey matter. \\
\textit{Other spatial features} Other features detected spatial smoothness. Signal components were expected to have high spatial smoothness, meaning a fairly small amount of connected clusters, where noise components were expected to have a more patchy spatial map. Features that extracted spatial smoothness in millimeters and voxels counts were therefore included. Some spatial features detected striped patterns of interleaved positive and negative activation. A largely striped pattern in the spatial map could imply that a component contained noise. \cite{Salimi-Khorshidi2014}

%The derived features from one heat run from one subject consist of a 25 x 186 matrix - one feature value for each feature of each component.



