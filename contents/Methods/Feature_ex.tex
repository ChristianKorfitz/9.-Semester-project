\section{Feature extraction}
An important step in establishing a robust classification model is extracting effective features to feed the classifier. Extraction of reasonably independent features that correlate with the targeted class will establish a more robust classifier when training it. In the FIX algorithm 186 features with temporal or spatial characteristics are extracted. The first feature extracted is, however, considered of spatial-temporal origin and consists of the number of IC computed by MELODIC. It will be expected that the noise present in the data will affect the number of ICs extracted. \cite{Salimi-Khorshidi2014} However, in this study the number of IC calculated for each heat run was fixed on 25, and that feature would therefore have no influence on the expectation of a component containing noise or signal. 
The following sections aim to explain the idea behind extracting these exact temporal and spatial features and to give examples of selected features. Thus, not all 186 features will be covered in depth. For a further explanation of the features we refer the reader to the 2014 study by Salimi-Khorshidi et al. \cite{Salimi-Khorshidi2014} The result of the feature extraction for one from one heat run from one subject consisted of a 25 x 186 matrix - one feature value for each feature of each component.

\subsection{Temporal features}
\textit{Autoregressive models} Temporal features extracted were based on autoregressive (AR) properties of the time course. It was expected that the temporal smoothness derived from AR models, could help distinguish signal from particular artefact components. For an AR(n) model, where the $n$ is the order up to $n = 6$, and $a_k$ denotes the AR parameters. Let $e_p$ denote the variance of the residual of up to AR(6). The first AR property features derived are the slope and intercept of the straight line which is explaining $e_p$ as a function of $n$, where an increase in $n$ will output a better fit and thus a smaller residual variance. It is expected that an improvement in goodness of fit will decrease as more noise is present in the time course. 
The other AR features extracted are simply $a_1$, $a_2$, $e_1$ and $e_2$. These features encapsulate the autocorrelation estimated from the low order AR models. It was expected that components containing signal would have a higher temporal autocorrelation and lower residual variance compared to components containing unstructured noise. \\
\textit{Distribution information} It is expected that signal components would be normally distributed, while noise components would vary in how they are distributed (e.g. long-tailed distribution and multi-modal), e.g. due to sudden spikes in the time course caused by rapid movements and/or scanner artefacts. Features that give information on distribution were therefore extracted, and these include kurtosis, skewness, mean-median difference, entropy and negentropy. \\
\textit{Jump amplitudes} The extent of jumps in the amplitude of the time course is an important feature in distinguishing signal from noise. Signal components are expected to be fairly smooth, while noise components usually will contain large jump and fluctuations. To describe this characteristic features based on division calculations that include the standard deviation (std), mean or maximum of the differential of the time course and the std, mean or maximum of the time course were extracted. \\
\textit{Fourier transform} Taking the fast Fourier transform (fft) of the time course will capture the frequency content, which can be utilized to differentiate between signal and noise. It is expected for signal components to almost exclusively have high power amplitude in low frequencies due to the block design of the stimuli task. Noise components may on the other hand contain frequency in the whole spectrum. Therefore several fft-based features were extracted. Examples of features derived were the total power above 0.1, 0.15, 0.2 and 0.25 Hz respectively and the percentage of power that lied in the following intervals: 0:0.01, 0.01:0.025, 0.025:0.05, 0.05:0.1, 0.1:0.15, 0.15:0.2 and 0.2:0.25 Hz. \\
\textit{Correlation} It is expected that time courses of signal components are strongly associated with grey matter (GM), while noise components will correlate with time courses of white matter (WM) and cerebrospinal fluid (CSF) along with the motion correction time courses calculated in the initial pre-processing. The last temporal features extracted were therefore based on correlation between the component time course and reference GM-, WM-, CSF-derived time courses and the motion correction time courses. The reference time courses were calculated using FSL’s tissue segmentation tool, from which GM, WM and CSF masks were extracted. Each type of tissue’s time course was then computed as the average of the time courses that correspond to that given tissue. \cite{Salimi-Khorshidi2014}

\subsection{Spatial features}
\textit{Clusters’ size and spatial distribution} The distribution of activated and deactivated cluster sizes is an important indicator of a component containing either noise or signal. It is expected that signal components contain small number of relatively large clusters, while noise components contain a large number of small clusters. To capture this information, a list of a spatial map’s clusters, $\mathbf{c}$ was formed, where only cluster of at least 5 connected voxels are kept, and listed in descending order. Feature extracted to summarize $\mathbf{c}$ are: length($\mathbf{c}$), mean($\mathbf{c}$), median($\mathbf{c}$), max($\mathbf{c}$), var($\mathbf{c}$), skewness($\mathbf{c}$), kurtosis($\mathbf{c}$), $\mathbf{c}$[1], $\mathbf{c}$[2], and $\mathbf{c}$[3], where the last three features were the first to third elements of $\mathbf{c}$.
Looking at the distribution of clusters in individual slices can help detect scanner artefacts. Let $\mathbf{V}$ and $\mathbf{U}$ contain slice specific information of the ICA spatial map $\mathbf{m}$ and slice specific information of voxels above 2.5 z-score $\mathbf{m^{\tau}_{p}}$, respectively. Further features consist of max($\mathbf{V}$) and max($\mathbf{u}$) and slices containing above 15 % of the total variance of $\mathbf{V}$ and $\mathbf{u}$. Similar features examines the difference in variance explained by even and odd slices as well and the difference in variance explained by slices $[1,2,5,6,9,10, …]$ and $[3,4,7,8,11,12, …]$ for both $\mathbf{m}$ and $\mathbf{m^{\tau}_{p}}$. 
It is not very likely for signal components to have large contents of both activation and deactivation in the spatial map. For this reason features that use the mean, standard deviation and entropy of $\mathbf{m}$ to measure the amount of positive and negative voxels in the components were extracted. \\
\textit{Voxels overlaying dark/bright raw data voxels}
As mentioned signal of interest is associated with dark voxels where noise is usually found in bright voxels. The next features are based on multiplying and dividing the components’ spatial map with the mean of the corresponding pre-processed time courses, forming two new images. It is expected that the intensity information contains information on which voxels are of interest and which are not. The features consists of the 95th and 99th percentile of the new images. \\
\textit{Percent on brain boundary} High Activation/deactivation in the spatial map that overlaps the boundary of the brain and the area outside the brain will most likely be movement related. Segmenting the brain using the FSL BET tool and subtracting this mask with and eroded version of itself will output a mask of the brain’s edge. Five masks with varying thickness are extracted. The features measure how large a percentage of the components spatial maps that is imbedded in the edges and how large a percentage of edges is covered by the component. The higher these values are, the higher probability the component was to have captured a movement artefact. \\
\textit{Mask-based features} It might be needed to use spatially-specific masks to detect certain noise sources (Sagital sinus, CSF and WM) more precisely. The spatial appearance of the major veins could for instance be mistaken for signal of interest and the cluster-like activation pattern may look similar to signal of interest. To detect these structures more conservatively FIX utilizes three standard-space masks that consist of three major veins. From the three masks three different masks with varying thickness are derived, due to the anatomical variability between subjects. The same features as in the brain boundary features are extracted, where these new masks were utilized instead. The same procedure was executed for GM. \\
\textit{Other spatial features} Other features detect spatial smoothness. Signal components are expected to have high spatial smoothness, meaning a fairly small amount of connected clusters, where noise components are expected to have a more patchy spatial map. Features that extract spatial smoothness in mm and voxels counts are therefore included. The last spatial features detect striped patterns of interleaved positive and negative activation. A largely striped pattern in the spatial map could imply that a component contains noise. \cite{Salimi-Khorshidi2014}

%The derived features from one heat run from one subject consist of a 25 x 186 matrix - one feature value for each feature of each component.



