\section{Brain Extraction and Registration} \label{BET}

In order to study the activation in the brain, it first has to be segmented and extracted from the surrounding tissue. To achieve an image only containing the brain, there were made use of the Brain Extraction Tool (BET), which is included in the FSL software package. As this is a standard applied method and not essential for the scope of this project, no further documentation of this step will be presented. For further documentation of BET specifications see Smith et al. \cite{Smith2002}. BET was used for extracting the brain in both the structural T$_1$-weighted image and in the functional T$_{2}^*$-weighted image sequence.    \\
The next step of preparing the functional images for analysis is to obtain the same coordinate system for these and the structural images. This step is called registration, and is done to find out more about the origin of the activation, in the form of spatial localization. As the functional images contain very low resolution activation can hardly be assessed and therefore activation can be studied by comparing the functional with a structural. This can either be achieved by registering the functional to the structural of the subject or to a standard template. The standard is made from an average of hundreds of brain. The template is useful as the spatial localization can be interpreted and reported objectively and consistently across studies. \cite{Hajnal2001}\\
For registration in this project, first, a registration of functional to structural has been made. This was achieved by using the FSL software tool FLIRT, which uses non linear transformations. Afterwards a registration from structural to the MNI template has been made. This process utilized both non linear and linear transformations by using the FLIRT and FNIRT tool in the FSL toolbox. \cite{Andersson2007,Jenkinson2001}
