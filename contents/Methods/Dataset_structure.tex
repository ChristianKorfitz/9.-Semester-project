\section{Dataset Structuring}

As presented earlier the total number of subjects included in this study was 139. Each subject underwent 3 heat runs where, for each, a functional scan was acquired, summited to a total of 417 scans. In \secref{art} it was introduced that the FIX preprocessing method utilizes a classification algorithm for separating noise sources from signal sources. The FIX software package comes with predefined training sets for training the classifier, but none of these have been made from fMRI scans consisting of signal related to the pain response from a noxious stimulus block design. Therefore, it was chosen to split the total dataset into a training dataset for training the classifier for this specific application and a test dataset for evaluating the performance of both preprocessing methods. \\
The training data set consisted of 34 subjects making a total of 102 scans. However, during the initial analyses two heat scans were found to have missing data resulting in the exclusion of these. This meant that the test dataset consisted of the remaining 105 subjects. However four subjects were found to not having the cerebellum fully scanned, which resulted in additional exclusion of these four subjects. In the remaining test dataset 7 scans were found to have a missing heat run scan.  Therefore, the test dataset consisted of a total of 296 scans.  \\
Approximately one fourth of the entire dataset was to be used as training data. In Salimi-Khorshidi et al. \cite{Salimi-Khorshidi2014} it was stated that a minimum of 10 subjects were required to train the classifier. With consideration to the hypothesized higher complexity of the acquired data, a higher number of subjects was implemented. However, it was prioritized that the data comprised of more than 100 subjects to retain high statistical power.
%
% \fxnote{regarding the inception of the rating periods of the scan, which resulted in the exclusion of these two heat scans. Furthermore, four subjects were found to not having the cerebellum fully scanned, which resulted in additional exclusion of these four subjects. }