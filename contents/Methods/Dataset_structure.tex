\section{Dataset Structuring}

As presented earlier the total number of subjects included in this study was 139. Each subject underwent 3 heat runs where, for each, a functional scan was acquired, summited to a total of 417 scans. In \secref{art} it was introduced that the FIX preprocessing method utilizes a classification algorithm for separating noise sources from signal sources. The FIX software package comes with predefined training sets for training the classifier, but none of these have been made from fMRI scans consisting of signal related to the pain response from a noxious stimulus block design. Therefore, it was chosen to split the total dataset into a training dataset for training the classifier for this specific application and a test dataset for evaluating the performance of both preprocessing methods. \\
The training data set consisted of 34 subjects making a total of 102 scans\fxnote{ELABORATE ON THIS BEING SUFFICIENT, INSERT CITATION, \cite{Salimi-Khorshidi2014} suggest that a minimum of 10 subjects}. However, during the initial analyses two heat scans were found to have missing data resulting in the exclusion of these. This meant that the test dataset consisted of the remaining 105 subjects. In the test dataset 7 scans were found to have missing information or incomplete data, resulting in the exclusion of these. Therefore, the test dataset consisted of a total of 308 scans.  

 