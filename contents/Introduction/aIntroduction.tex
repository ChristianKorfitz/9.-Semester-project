To experience and live with chronic pain can be immensely debilitating for any individual, and the consequences of experiencing chronic pain has been linked to numerous physical and mental conditions: restrictions in mobility and daily activities, dependency of therapeutic drugs, anxiety, depression and a reduction in quality of life \cite{Dahlhamer2018,NationalCenterforHealthStatisticsHealth2011}. \\ More than 100 million Americans are estimated to live with chronic pain, manifesting an extraordinary amount of human suffering along with additional large economic societal expenditures, as cost of medical care and loss of wages and productivity are escalating  \cite{InstituteofMedicine2011,Davis2017}. Compared to other major health conditions the number of patients suffering from chronic pain outnumber the total of patients suffering from heart disease and stroke, diabetes and cancer, and these conditions often include pain as a contributing component \cite{NationalCenterforHealthStatisticsHealth2011}. \\
“An unpleasant sensory and emotional experience associated with actual or potential tissue damage, or described in terms of such damage”, is how the International Association for the Study of Pain defines pain \cite{Merskey1994}. The phenomenon of pain is a very complex composition of both psychological factors, personality traits and states, and cognitive, emotional, motivational, contextual and cultural variables. Pain can furthermore be defined as either acute or chronic. Acute pain is a healthy response making up a self defense mechanism to warn and protect the human organism from any potential or further harm. \cite{Davis2017,Brook2011,Garland2013} Contrary to acute pain, chronic pain is not linked to organ damage of any kind, and does therefore not serve any useful purpose to the human organism \cite{Schmidt1986}. \\
It still remains incredibly challenging to assess and treat pain, as the mechanisms causing the experience of pain are not yet fully understood \cite{Nielsen2008,Coghill2011}. This includes all neurological steps in the processing of pain - from the peripheral detection of a stimulus to the spinal cord transmission and brain processing, including the further descending relay of the pain experience \cite{Feizerfan2015}. A step towards getting a higher understanding of chronic pain is to examine the mechanisms involved in acute pain. A justification for this reasoning is that chronic pain often is transitioned from repetitive acute nociceptive stimulations that lead to neuroplastic changes. \cite{Feizerfan2015, Mcgreevy2012}\\
Multiple studies \cite{Davis2017,Coghill2003,Kim2004,Emerson2014} have shown that the perception and sensitivity to acute pain are disposed to a great extent of subjective variability, suggesting that a pivotal key in understating the brain mechanism during noxious stimuli is found in the individual differences between subjects. Additionally, an individual's sensitivity can vary substantially from day to day, despite being exposed to the same stimuli. In some instances the psychophysical rating reported by the patient can rather be seen as a bias of the various scales used for reporting, than actually reflecting an unbiased measure of the experienced pain. It can furthermore be difficult for the physician to evaluate the patient's experienced pain from a third-person perspective. \cite{Coghill2003} \\
To examine if there is an actual difference between individuals in brain activation during the experience of pain and if this activation is correlated with the self-reported experience of pain could provide useful insight in the study of pain. 