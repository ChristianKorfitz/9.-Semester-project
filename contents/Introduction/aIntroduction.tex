
To experience and live with pain can be immensely debilitating for any individual, and the consequences of experiencing pain has been linked to numerous physical and mental conditions: restrictions in mobility and daily activities, dependency of therapeutic drugs, anxiety and depression and a reduction in quality of life \cite{Dahlhamer2018,NationalCenterforHealthStatisticsHealth2011}. \\ More than 100 million Americans are estimated to live with pain, manifesting an extraordinary amount of human suffering along with additional large economic societal expenditures \cite{InstituteofMedicine2011}. Compared to other major health conditions pain outnumber the combined total of heart disease and stroke, diabetes, and cancer combined \cite{NationalCenterforHealthStatisticsHealth2011}. \\
“An unpleasant sensory and emotional experience associated with actual or potential tissue damage, or described in terms of such damage”, is how the International Association for the Study of Pain defines pain \cite{Merskey1994}. The phenomena of pain is a very complex composition of both sensory, reactive and cognitive components making up a self defense mechanism to warn and protect the human organism from any potential or further harm \cite{Brook2011,Garland2013}. It still remains incredibly challenging to define and treat pain, as the brain mechanisms during the experience of pain is not yet fully understood \cite{Nielsen2008,Coghill2011}. \\
Multiple studies \cite{Coghill2003,Kim2004,Emerson2014} have shown the perception and sensitivity to pain are disposed to a great extent of subjective variability, suggesting that a pivotal key in understating the brain mechanism during pain is found in the individual differences between subjects. Additionally, individual's sensitivity can vary substantially from day to day, despite being opposed to the same stimuli, and the psychophysical rating given by the patient is rather looked upon as artifact of scale, than actually reflecting an unbiased measure of the experienced pain \cite{Coghill2003}. It can furthermore be difficult for the physician to evaluate the patient's experienced pain from a third-person perspective. To examine if there is an objective measure of pain sensitivity and if it then correlates with the reported experience is therefore of great interest \cite{Coghill2003}. \\ 
A potential objective measure of pain sensitivity could be utilizing the method of functional Magnetic Resonance Imaging (fMRI), as it depicts the physiological brain response to noxious stimulus. A prior study \cite{Coghill2003} demonstrated that individuals reporting different pain sensitivity when exposed to the same stimuli showed correlated brain activation in areas of SI, ACC and prefontal cortex .

Transition to our problem at hand

End up with something presenting our comparison of preprocessing method and this work 
would contribute to the scientific community.  






