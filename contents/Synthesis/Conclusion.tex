\chapter{Conclusion}

The project had the following project: \textit{Evaluate the performance of using the standard preprocessing pipeline compared to incorporating FSL FIX in the pipeline for noise removal in images acquired with a BOLD fMRI sequence in the context of individual cerebral activation as response to noxious heat stimuli.} The main aim of the study was to investigate if using the automatic denoising tool FIX would provide a better noise cleanup than using standard preprocessing. From the result, it was seen that areas of interest in the likes of the insular cortex, anterior cingulate cortex and basal ganglia, had a significantly higher activation intensity (Z-score>2.3), when using the FIX pipeline for preprocessing compared to standard preprocessing. The use of FIX did generally not seem to affect the localization of brain activation, but resulted in a higher signal intensity preservation in the same areas detected by using the standard preprocessing. \\
The secondary research aim was to investigate whether there could be found a relation between individual’s pain sensitivity and their corresponding brain activation when using FIX preprocessing compared to using standard preprocessing. It was found that using standard preprocessing did not elucidate any individual differences. The same result was found when employing FIX for preprocessing. Thereby, FIX did not have a significant impact on the detecting of individual differences. The hypothesis regarding the existence of a relation between pain sensitivity and brain activation can therefore not be either confirm or disproved following the results of this project. Thus, further research on the hypothesis is still needed. 