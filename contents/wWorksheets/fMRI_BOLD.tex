\section{Functional MRI}

Top stating something about why fmri and only bold 

\subsection{Introduction to fMRI}
Functional magnetic resonance imaging measure the changes in deoxyhemoglobin concentration associated with different neurological tasks. The changes in concentration show the metabolism of the different brain areas. fMRI offers many great advantages, which has made it a very popular method for imaging brain activity, where some advantages are high temporal and spatial resolution, low cost, and most of all it being non-invasive. The versatility of fMRI has made it very important tool which is used in a widespread of tasks like being a biomarker for diseases and study the efficacy of pharmaceuticals.\cite{Glover2011}

Brain metabolism that facilitate the possibility of fMRI: 
Multiple steps in forming and transmitting a neurological signal requires energy ATP consumption, like reception and reformation of an action potential. When activating a brain area as done by e.g. finger tapping, the ATP starts to be processed, this results in a decrease in oxygen concentration and increase in waste build. Thereby the metabolic need for oxygen increases. These factors, which are present in the local tissue, activate a vasodilation increasing the blood flow to that area. Homeostasis is thereby reestablished. Though one special and not fully understood thing happens during this process. More oxygenated blood than needed to deal with the offset is delivered to the specific area. As a result, neural up-regulation results initially in a build-up of deoxygenated hemoglobin [Hb] and a decrease in deoxygenated hemoglobin [HbO2] in the intra- and extravascular spaces, followed within a second or two by a vasodilatory response that reverses the situation to result in an increase in [HbO2] and decrease in [Hb] over that in the resting condition. The increase in neural activity in that specific area thereby permits two conditions which can be assessed by fMRI, being the cerebral blood flow and blood oxygen level dependent contrast.\cite{Glover2011}

\subsection{BOLD imaging}
The crucial part of this in relation to doing BOLD imaging is that blood fully oxygenated, HbO2 is diamagnetic and is magnetically indistinguishable from brain tissue. However, fully deoxygenated Hb has 4 unpaired electrons and is highly paramagnetic.
Thereby more oxygenated blood in the area the larger the contrast seen.  
\cite{Glover2011}
BOLD MRI kan aabenbart uddybes mere.. Se handbook k.5 for mere materiale

Strengths and weaknesses: All of the resolution mentioned above as well as being non-invasive. It can provide high resolution of anatomical structures and localization and visualization of vessels. Furthermore it can provide information of white matter connectivity through Diffusion Tensor Imaging. The downside is BOLD signals are were sensible to noise as the original signal is often lower than the noise. Additionally the loud scanner noise may distract patients during fMRI scans. 
\cite{Glover2011}

Major factors when doing fMRI:
The analysis of fMRI data is made complex by a number of factors. First, the data are liable to a number of artifacts, such as those caused by head movement. Second, there are a number of sources of variability in the data, including variability between individuals and variability across time within individuals. Third, the dimensionality of the data is very large, which causes a number of challenges in comparison to the small datasets that many scientists are accustomed to working with. The major components of fMRI analysis are meant to deal with each of these problems. \cite{Glover2011} 
