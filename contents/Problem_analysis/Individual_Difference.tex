\section{Individual differences in pain perception and fMRI}
The study of individual differences is the investigation of an individual’s response to a similar intervention. This has been of great interest in medical fields as understanding individual pain experience would shape more precise treatments given the individual’s unique characteristics. However, understanding why individuals react differently across sensory modalities has for long been a perplexing problem. In pain research, studies have shown a large variability in individual pain sensitivity as reported on a Visual Analogue Scale (VAS), when exposed to identical noxious stimuli \cite{Nielsen2008, Coghill2003}. This could indicate that individuals have different neurological responses to the same noxious stimuli, but on the other hand raise questions about the individual's integrity: is the subject over/underplaying the pain experience or engaging in drug seeking behavior? Thus, it has been of great interest to examine if the subjectively reported pain experience correlates with the actual neurological response in the individual. \cite{Coghill2011} Verifying a correlation between self-reports and brain activity during evoked pain stimulus would function as an important step in improving physician’s understanding of the mechanisms underlying individual differences in pain perception and the treatment of patients suffering from chronic pain. Brain imaging could on sight be an assistive tool to self-reports and behavioral evidence when assessing a patient’s claim regarding pain and physical condition. Such an addition would serve an especially valuable function when assessing patients unable to communicate verbally (e.g. young children or patients with dementia), patients with self-reports and behavioral evidence that are conflicting and for personalized pain management. In perspective, as brain imaging gets more accepted as an assistive tool in personalized pain management it will gain more interest in legal practices. Using it as proof or disproof on whether a patient is actually experiencing pain might affect financial outcome in insurance cases. However, due to ethical and jurisdictive circumstances it would be inappropriate to use it as an assistive tool in such cases until it has been sufficiently verified. \cite{Davis2017} \\
An indirect measure of functional brain response can be achieved through a BOLD fMRI scanning, and is often used to get a generic understanding of brain function in research. Averaging data across individuals is commonly applied to raise the signal-to-noise ratio (SNR). However, while providing a general knowledge on brain function, this practice excludes the possibility of observing brain function in the individual. The interest of examining individual differences in brain activity has been present for several years, but the technology has only been advanced enough during recent years for it to be carried out, due to higher magnetic field strength and faster acquisition time. \cite{Dubois2016} \\
%In 1999 and 2003 Coghill et al. \cite{Coghill1999, Coghill2003} published studies showing correlation between intensity of activation of specific brain regions associated with processing of noxious stimuli and subjects’ subjective pain sensitivity. Positron Emission Tomography (PET) scans and BOLD fMRI in a 1.5 tesla scanner was used respectively in the two studies for functional brain mapping. 
%However, as Dubois et al. \cite{Dubois2016} emphasize only during the recent years the MRI technology has been advanced enough to create images with sufficient SNR to use as means in studying individual differences. A review study by Wood et al. \cite{Wood2012} comparing 1.5 tesla and 3.0 tesla scanners, showed that using the BOLD fMRI technique can produce better susceptibility contrast sensitivity and due to the naturally higher SNR in 3.0 tesla scanners a 40 percentage increase in activation detection can be generated. \\
%In that relation, a validation of the 2003 study by Coghill et al. \cite{Coghill2003} with data from a 3.0 tesla scanner would be relevant, as a 1.5 tesla scanner was used for BOLD fMRI acquisition in \cite{Coghill2003}. Further reasoning for a validation study is that clinics tend to get higher tesla scanners installed, when desiring to get finer images in fMRI \cite{Wood2012}. Thus, the future of fMRI lies within the use of higher tesla scanners. In \cite{Coghill2003}, a total number of 17 subjects were included. To raise the reproducibility and verification in a validation study the number of participants included should be increased to examine if statistically significant result can be found in a higher sample size \cite{Dubois2016, Button2013}. \\
However, a downside of 3.0 tesla scanners compared to 1.5 tesla scanners is that various artifacts related to motion, respiration, bloodflow, pulsation of cerebrospinal fluid and air-tissue interfaces are more prominent \cite{Wood2012}. In that relation it would be beneficial to identify which concrete artifacts that are associated with BOLD fMRI along with which methods that exist to de-noise the acquired images.   
%In that relation, examining different types of preprocessing methods to denoise the images would be of high interest. 
