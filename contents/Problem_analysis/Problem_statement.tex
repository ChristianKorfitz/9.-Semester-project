\chapter{Study Objective}

In summary it would be beneficial for the understanding of chronic pain to gain further knowledge on the brain mechanisms associated with acute pain. As different individuals report a wide variety in pain sensitivity when given identical pain stimuli, an examination on if the subjective reports correlate with intensity in brain activation in brain regions associated with pain would be of high interest. When using 3.0 tesla MRI scanners to indirectly measure brain activity through BOLD fMRI sequences, more of the activity will be captured compared to using 1.5 tesla scanners. The images will on the other hand be more prone to contain noise. For this reason proper preprocessing is needed to separate noise from signal of interest before analysing the images. During the recent years more preprocessing methods have been developed in the field of fMRI as alternatives to the standard method. These methods include the CompCor and the FSL FIX. A comparison of these newer methods and the standard method on a large set of subjects would be favorable in the field of pain research using fMRI, as this could provide insight on the advantages/disadvantages of each preprocessing method. This introduces the study objective:

\begin{center}

\textit{Compare and evaluate the performance of standard preprocessing, CompCor and FSL FIX for noise removal in images acquired with BOLD fMRI studying individual subjects' brain activation to noxious heat stimuli.}

\end{center}