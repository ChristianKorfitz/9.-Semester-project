
\section{The Hemodynamic Response}

As specified in the prior section, certain brain regions are activated when the body is subjected to a noxious stimuli. To further understand the regions activated during the experience of pain and how this activity is modulated by the brain, a measure of brain activation is therefore needed. To get a measure of brain activation, the underlaying physiologic response of a brain region activation is therefore of great importance to understand. 

The activation of a brain region starts with a neurological input containing information about the noxius stimuli \cite{Tracey2007}. The increased neurological activity effects local metabolism as processing of the signal requires adenosine triphosphate (ATP) consumption during e.g. the reception and reformation of the action potential. Thus, ATP starts to be processed, leading to a decrease in oxygen concentration and increase in waste. Thereby the metabolic need for oxygen increases. As the movement is planned and executed, factors, which are present in the local tissue of the corresponding brain area, activate a vasodilation, increasing the blood flow to that area to reestablish the local homeostasis. During this regulation a special and not fully understood phenomenon occurs as more oxygenated blood than needed to compensate for the offset is delivered. Thereby flooding the local region with oxygenated blood. The overall increase in neural activity in that specific area following the need for metabolic regulation thereby permits blood oxygen level dependent (BOLD) contrast to studied. An example illustrating the measurable hemodynamic response can se found in \figref{fig:back:HRF}. \cite{Glover2011,Poldrack2011}

\begin{figure}[H]                 
	\includegraphics[width=.48\textwidth]{figures/aBackground/HRF}  
	\caption{A depiction of a single hemodynamic response curve. ID is the initial dip as less oxygen will be present as the metabolic demand increases, TP is time from stimulus until peak, W and H is the width and height of the response and PSU is a post stimulus undershoot. \cite{Poldrack2011}}
	\label{fig:back:HRF} 
\end{figure}

\Figref{fig:back:HRF} should be considered the perfect noiseless hemodynamic response curve, to a brief stimuli, though the reality is that the response is noise and delayed in time. The peak height of the curve is commonly the most interesting feature of the response, as it portrays the amount of neural activity. The time to peak will occur 4-6 seconds after stimulus onset. The duration of a response is around 20 seconds. There will further be a noticeable initial dip of 1-2 seconds duration and a 20 second poststimulus undershoot. \cite{Poldrack2011}    \\  