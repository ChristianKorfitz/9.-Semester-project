\section{Artifacts Associated with BOLD fMRI} \label{sec:noise}

When BOLD fMRI sequences are used to detect brain activity during in vivo experiments, not only task related activation will be present. Unfortunately, in most experiments sources of noise will inevitably impact the scan. \cite{Salimi-Khorshidi2014} \\
%Artifacts that usually influence the scan and thus the signal of interest are motion, respiration, blood flow/heart beat, pulsation of cerebrospinal fluid, air-tissue interfaces and scanner inhomogeneities.
As BOLD fMRI relies on very precise temporal and spatial placement, even millisecond or millimeter fluctuations caused by motion can have a large impact on the quality of the acquired signal. Motion does not only distort the brain in space but also disrupts the formation of the magnetic gradients that enables the BOLD signal to be detected correctly. Bulk motion of the head will cause the content of each voxel to change. As the maximum value of net magnetization in the direction of the magnetic field (M$_0$) is directly proportional to the number of spins in that voxel the BOLD signal will consequently be altered. In tissue interfaces such as white/grey matter boundaries, at the edge of the brain and around large vessels this is especially an issue. Furthermore, bulk motion will change the uniformity of the magnetic field as is has been tuned for a particular head position. This also directly affects M$_0$ and results in dropouts, reductions in signal intensity, in the BOLD signal and thus in the subsequent readout. Lastly, movement will change steady state magnetization, by changing the time between excitations in the tissues that have moved from a slice to another. M$_0$ is affected until steady state has been reached again, and is referred to as the spin history effect. The spin history effect can cause the intensity of the BOLD signal to be detected as twice the level of the expected signal. \cite{Murphy2013} This will often result in an image where the intensities change in a striped pattern \cite{Poldrack2011}. \\
A typical protocol in pain research involves inducing noxious stimuli on the participant while recording the brain activity. These stimuli often causes the participant to move even more. During the movement some brain regions might also show activation associated with stimulus. Therefore it is easy to mistake brain activation with stimulus correlated movement when analyzing the data, resulting in a weaker or even false statistical analysis. \cite{Poldrack2011} \\
Motion due to cardiac and respiratory cycles will cause the same artifacts as during bulk motion of the head. Additionally, cardiac pulsation and respiratory cycles will cause the brain stem to push the surrounding brain tissue. This will cause deformation and movement of cerebrospinal fluid that will form changes in M$_0$. Further artifacts can be found related to the frequency content of cardiac and respiratory cycles. The frequency of cardiac and respiratory cycles at rest are around 1 Hz and 0.3 Hz respectively. Relatively high-frequent compared to the BOLD signal of < 0.1 Hz. As the repetition time in a BOLD fMRI usually is 2-3 seconds, parts of the cardiac and respiratory cycles will be aliased into the BOLD signal’s frequency bandwidth. Thus, these artifacts can be mistaken for the BOLD signal when examining the frequency content. Another respiratory factor that affects the BOLD signal is the arterial level of CO$_2$. It works as a vasodilator and facilitates a global increase in the cerebral blood flow and thereby the BOLD signal. Thus, when a participant holds the breath hypercapnia arises and the BOLD signal increases, and during hyperventilation hypocapnia arises causing a decrease in the BOLD signal.  \cite{Murphy2013} This type of artifact might occur more when the participant is being subjected to noxious stimuli to attenuate the pain experienced. \\
Furthermore, artifacts can form in areas where air and tissues meet, e.g. sinuses and the ear canals, and is caused by the main magnetic field inhomogeneities the air-tissue interface produce. It will be visualized as dropouts in the brain region adjacent to the air-tissue interface. \cite{Poldrack2011} \\
It can be summarized that many sources of noise can alter and disguise the signal of interest. Thus, being able to separate signal sources from noise is of great importance before analysing BOLD fMRI data, to avoid making any conclusions based on spurious data.



