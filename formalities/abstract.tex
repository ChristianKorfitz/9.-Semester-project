{\small More than 100 million Americans live with chronic pain; a highly debilitating condition. However, the mechanisms involved in the experience of chronic pain are not fully understood. Gaining more insight in acute pain might be a step towards expanding knowledge on chronic pain. fMRI can be used to image brain activity of participants receiving noxious stimuli. More clinics switch from using 1.5 tesla scanner to 3.0 tesla scanner. The use of 3.0 tesla scanners for fMRI improves brain activation detection, but simultaneously captures more physiological noise, compared to 1.5 tesla scanners. \\
FIX is a newly developed denoising algorithm that can be used to identify various noise components and denoise a large amount of data. It uses an ICA approach in identifying sources of signal in the data, which are used to train a classifier. The classifier can then identify sources in a larger data set, from which the noise sources can be removed. The primary aim for this project was therefore to evaluate FIX in comparison to standard preprocessing on a large data set acquired from a noxious stimuli protocol, to assess if FIX could generate a higher signal-to-noise ratio. \\
Using FIX resulted in a greater noise removal compared to standard preprocessing, leading to a greater amount of signal of interest related activation. Greater intensities of activation were mainly found in the areas of the insular cortex, anterior cingulate cortex and basal ganglia using FIX (Z-score>2.3). 
In conclusion, using the FIX preprocessing pipeline on task-related fMRI preserves a significantly higher amount of brain activation compared to conventional standard preprocessing.
\vspace{-0.5cm}}