To live with chronic pain can be immensely debilitating and often results in a lower quality of life for individual suffering from chronic pain. The experience of pain is subjective and can be seen a very complex composition multiple variables. FMRI is a modality which can be used to increase the knowledge of pain, where result found from studying acute pain can transferred to chronic pain. To further enhance the study of pain newer higher tesla MRI scanners are utilized introducing the possibility of more unwanted noise increasing the need for enhanced preprocessing methods. The aim of this study was to investigate the new method of FSL FIX for preprocessing of large datasets and compare the results to the conventional standard preprocessing. The study included fMRI scans from 139 subjects imaging the hemodynamic response to a noxious heat stimuli. The difference in brain activation intensity was assessed using the general linear model after applying both types of preprocessing to the datasets. Using FSL FIX resulted a greater noise removal compared to standard preprocessing, leading to a greater amount of signal of interest related activation. Greater intensities of activation were mainly found in the areas of the insular cortex, anterior cingulate cortex and basal ganglia using FSL FIX. In conclusion, using the FSL FIX preprocessing pipeline on task-related fMRI yields a significantly higher amount of brain activation (Z-score > 2.3) compared to conventional standard preprocessing. 